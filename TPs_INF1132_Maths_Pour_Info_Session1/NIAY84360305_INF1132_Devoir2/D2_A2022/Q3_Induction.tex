
\newcommand{\Pgcd}{{\rm Pgcd}}

{\textsc{\underline{Question 3 sur l'induction et la récursivité (\textbf{20} points)}}}\\


Les parties A et B sont \underline{indépendantes}.\\

\textbf{Partie A} (10 points) D\'efinitions inductives\\
    
Soit $\mathcal{B}$ un ensemble de cha{\^\i}nes binaires, défini récursivement de la mani\`ere suivante~:
\begin{enumerate}
    \item[$\bullet$] \textbf{Cas de base.} $\varepsilon \in \mathcal{B}$;  %0 \in \mathcal{B}; 1 \in \mathcal{B};$

    \item[$\bullet$] \textbf{R\`egles r\'ecursives.} Si la cha{\^\i}ne  $b \in \mathcal{B}$, alors,
    
    \begin{enumerate}[R1.]
     \item   $1b0 \in \mathcal{B} $  \quad\mbox{(r\`egle 1)}
     \item   $0b1 \in \mathcal{B}$  \quad\mbox{(r\`egle 2)} 
     \item  $bb \in \mathcal{B}$  \quad\mbox{     (r\`egle 3)}
    \end{enumerate}

\end{enumerate} 
\begin{enumerate}[\bf 1)]
\item\bareme{2}Parmi les cha{\^\i}nes binaires suivantes, lesquelles sont des \'el\'ements de $\mathcal{B}$?

$0010 1011$, $000000$, $0101$, $01010101$, $11001100$, $0$, $0110110$, $010101010101$

%%%%%%%%%%%%%%%%%%%%%%%%%%%%%%% 
% Solution                    %
%%%%%%%%%%%%%%%%%%%%%%%%%%%%%%%
\begin{framed}

RÉPONSE: $00101011 \in \mathcal{B}$ \\
        $0101 \in \mathcal{B}$ \\
        $01010101 \in \mathcal{B}$ \\
        $11001100 \in \mathcal{B}$ \\
        $010101010101 \in \mathcal{B}$ \\

\end{framed}
%%%%%%%%%%%%%%%%%%%%%%%%%%%%%%%

\item\bareme{2} Calculer toutes les cha{\^\i}nes de $\mathcal{B}$ de longueur $n<11$

%%%%%%%%%%%%%%%%%%%%%%%%%%%%%%% 
% Solution                    %
%%%%%%%%%%%%%%%%%%%%%%%%%%%%%%%
\begin{framed}

RÉPONSE: Je calcule les chaînes de $\mathcal{B}$ de longueur $n<11$ : \\
Si la chaîne $\mathcal{B}$ est de longueur n=2, le nombre de chaîne est de $2$\\
Les chaines de $B$ de longueur 2:\\ $01$ \\ $10$.\\
Si $\mathcal{B}$ est de longueur n=4, alors le nombre de chaîne est de $4$.\\
Les chaines de $B$ de longueur 4: \\ $0101$ \\ $0011$ \\ $1010$ \\ $1100$\\
Si $\mathcal{B}$ est de longueur n=6, alors le nombre de chaîne est de $8$.\\
Les chaines de $B$ de longueur 6: \\ $001011$ \\ $001101$ \\ $010101$ \\ $010011$\\ $101010$ \\ $101100$ \\ $110100$ \\ $110010$\\
Si $\mathcal{B}$ est de longueur n=8, alors le nombre de chaîne est de $16$.\\
Les chaines de $B$ de longueur 8: \\ $00101011$ \\ $00101101$ \\ $00110011$ \\ $00110101$\\ $01001011$ \\ $01001101$ \\ $01010011$ \\ $01010101$\\ $10101010$ \\ $10101100$ \\ $10110010$ \\ $10110100$\\ $11001010$ \\ $11001100$ \\ $11010010$ \\ $11010100$\\
Si $\mathcal{B}$ est de longueur n=10, alors le nombre de chaîne est de $32$.\\
Les chaines de $B$ de longueur 10:\\ $0101010101$ \\ $0101010011$ \\
$0101001101$ \\ $0101001011$ \\ $0100110101$ \\ $010011001 1$ \\ $0100101101$ \\ $0100101011$ \\ $0011010101$ \\ $0011010011$ \\ $0011001101$ \\ $0011001011$ \\ $0010110101$ \\ $0010110011$ \\ $0010101101$ \\ $0010101011$ \\ $1101010100$ \\ $1101010010$ \\
$1101001100$ \\ $1101001010$ \\ $1100110100$ \\ $1100110010$ \\ $1100101100$ \\ $1100101010$ \\ $1011010100$ \\ $1011010010$ \\ $1011001100$ \\ $1011001010$ \\ $1010110100$ \\ $1010110010$ \\ $1010101100$ \\ $1010101010$
\end{framed}

%%%%%%%%%%%%%%%%%%%%%%%%%%%%%%%

\item\bareme{3} Montrer que toutes les cha{\^\i}nes  de $\mathcal{B}$ ont un nombre \'egal de $0$ et de $1$.

%%%%%%%%%%%%%%%%%%%%%%%%%%%%%%% 
% Solution                    %
%%%%%%%%%%%%%%%%%%%%%%%%%%%%%%%
\begin{framed}

RÉPONSE: Les chaînes de $\mathcal{B}$ sont des chaînes binaires. On obtient les nombres binaires à partir du reste de la division euclidiènne d'un nombre par $2$, or il n'y a que deux restes possibles après une division par $2$, ($0$ ou $1$).
Alors on peut conclure que toutes les chaînes binaires sont composées que de $0$ et de $1$, étant donné que $\mathcal{B}$ est une chaine binaire alors elle est compte que des $0$ et des $1$.

\end{framed}
%%%%%%%%%%%%%%%%%%%%%%%%%%%%%%%
\item\bareme{3} Montrer que toutes les cha{\^\i}nes  de $\mathcal{B}$ sont de longueur paire
%%%%%%%%%%%%%%%%%%%%%%%%%%%%%%% 
% Solution                    %
%%%%%%%%%%%%%%%%%%%%%%%%%%%%%%%
\begin{framed}

RÉPONSE:
$\mathcal{B}$ est une chaîne binaire qui repond aux règles suivants:\\ 
$1b0 \in \mathcal{B} $, $0b1 \in \mathcal{B} $ et $bb \in \mathcal{B} $\\
$b$ est une séquence binaire qui peut-être une chaîne vide ou peut prendre plusieurs chaînes de longueur $2$ ($01$ ou $10$).\\
Vu cette configuration est peut importe le nombre de $b$ ajouté, on obtiendra une chaine de longueur paire car le produit de tout entier par $2$ sera pair. \\
Donc toutes les chaînes de $\mathcal{B}$ sont de longueur paire.

\end{framed}
\end{enumerate}

\newpage
\textbf{Partie B} (10 points) Algorithmes r\'ecursifs\\

Pour calculer le plus grand commun diviseur de deux entiers positifs on peut utiliser
l'algorithme suivant:\\
\rule{0.8\textwidth}{0.4mm}
            \begin{algorithmic}[1]
                \Function{\Pgcd}{a,b: Entiers} 
                \If {$(a\bmod b = 0)$} 
                    \State \Return{($b$)}
                \Else 
                    \State \Return{$\Pgcd(b, a\bmod b)$}
                 \EndIf
                \EndFunction
            \end{algorithmic}
\rule{0.8\textwidth}{0.4mm}


\begin{enumerate}[\bf 1)]
\item\bareme{5}  Faites la trace d'ex\'ecution  de cet algorithme sur l'appel
  $$\Pgcd(420,567)$$

\begin{framed}

RÉPONSE:\\
\setcounter{MaxMatrixCols}{20}
\[
\begin{matrix}
 Nbre Appel&\vline&Pgcd \\
\hline
1&\vline&(567,420)  \\
2&\vline&(420,147)  \\
3&\vline&(147,126)  \\
4&\vline&(126,21)=21  \\
\end{matrix}
\]

\end{framed}

\item\bareme{5} D\'eterminez le nombre d'appels effectu\'es en fonction de $n$ dans le cas des appels suivants:
$$\Pgcd(F_{n - 1},F_n) \quad \mbox{et} \quad \Pgcd(F_n, F_{n-1}),$$
o\`u $F_n$ est la suite d\'efinie par
$\
F_0=0 ; F_1=1;
F_n=F_{n- 1}+F_{n-2}$ si $n\geq 2$.
Pour vous donner une id\'ee, essayez de calculer $\Pgcd(34,55)$ et $\Pgcd(55,34)$.



\begin{framed}

RÉPONSE:\\
Je determine le nombre d'appels effectués en fonction de n:

je calcule d'abord $\Pgcd(34,55)$
\setcounter{MaxMatrixCols}{20}
\[
\begin{matrix}
 Nbre Appel&\vline&Pgcd \\
\hline
1&\vline&(34,55)  \\
2&\vline&(55,34)  \\
3&\vline&(34,21)  \\
4&\vline&(21,13)  \\
5&\vline&(13,8)\\
6&\vline&(8,5)\\
7&\vline&(5,3)\\
\end{matrix}
\]
\[
\begin{matrix}
8&\vline&(3,2)\\
9&\vline&(2,1)=1\\
\end{matrix}
\]

je calcule d'abord $\Pgcd(55,34)$
\setcounter{MaxMatrixCols}{20}
\[
\begin{matrix}
 Nbre Appel&\vline&Pgcd \\
\hline
1&\vline&(55,34)  \\
2&\vline&(34,21)  \\
3&\vline&(21,13)  \\
4&\vline&(13,8)\\
5&\vline&(8,5)\\
6&\vline&(5,3)\\
7&\vline&(3,2)\\
8&\vline&(2,1)=1\\
\end{matrix}
\]
Conclusion: On peut en-conclure que :\\
Pour trouver le Pgcd($F_{n-1},F_n)$ on fait $n-1$ appels.\\
Pour trouver le Pgcd($F_n,F_{n-1})$ on fait $n-2$ appels.

\end{framed}
\end{enumerate}


